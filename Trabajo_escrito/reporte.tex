\documentclass[12pt]{article}
\usepackage{enumitem}

\usepackage[left=25mm, right=25mm, top=25mm, bottom=25mm, headheight=25mm]{geometry}
\usepackage{graphicx} % Required for inserting images
\usepackage{subfig} % Poder colocar más de una imagen en una {Figura}
\usepackage[figurename=Imagen]{caption} % Cambiar la descripción de la Image 
\usepackage[utf8]{inputenc} % Idioma español
\usepackage[spanish]{babel} % Idioma español

% Para poder ajustar los colores en las referencias que se hagan
\usepackage{hyperref}
\hypersetup{
    colorlinks=true,
    linkcolor = black,
    citecolor = blue,
    urlcolor = blue
}

% Dar formato a las páginas y colocar el \rfoot{De forma que nos interese}
\usepackage{parskip}
\usepackage{lastpage}
\usepackage{xcolor}
\usepackage{fancyhdr}
\pagestyle{fancy}
\cfoot{}
\rfoot{Página \thepage\ de \pageref{LastPage}}
\rhead{}
\lhead{}
\renewcommand{\headrulewidth}{0pt}

% Parámetros que pueden cambiar ----------------------------------------------------------------------------
% Nombres
\newcommand{\Profesor}{Profesor: Edgar Tista García}
\newcommand{\Materia}{Asignatura: Estructura de Datos y Algoritmos II}
\newcommand{\Grupo}{Grupo:  5}
\newcommand{\Nombre}{Integrantes: , Vázquez Flores José Angel, Zarate Menes Quetzalli}
\newcommand{\NoLista}{No. de lista: , 35}
\newcommand{\Proyecto}{PROYECTO \#2 - ÁRBOLES BINARIOS }
\newcommand{\Semestre}{Semestre: 2024-2}
\newcommand{\Fecha}{Fecha de entrega: 25 de mayo del 2024}
\newcommand{\Observaciones}{Observaciones:}
\newcommand{\Calificacion}{CALIFICACIÓN:}

\newcommand{\TituloP}{TRABAJO ESCRITO DEL PROYECTO }
\newcommand{\Titulo}{\Proyecto\\\TituloP}


\begin{document}
\begin{center}
    \LARGE\textbf{\TituloP}\par\vspace{0.7cm}
\end{center}
\Large{\Proyecto}\par\vspace{0.6cm}
\Large{\Profesor}\par\vspace{0.6cm}
\Large{\Materia}\par\vspace{0.6cm}
\Large{\Grupo}\par\vspace{0.6cm}
\Large{\Nombre}\par\vspace{0.6cm}
\Large{\NoLista}\par\vspace{0.6cm}
\Large{\Semestre}\par\vspace{0.6cm}
\Large{\Fecha}\par\vspace{0.6cm}
\Large{\Observaciones}\par\vspace{0.6cm}

\begin{center}
    \Large{\Calificacion}\par\vspace{0.6cm}
\end{center}
% -------------------------------------------------------- Empieza el desarrollo de la práctica

\newpage
\section*{\textcolor{red}{\textbf{OBJETIVO}}}
\par\vspace{0.4cm}
Que el alumno implemente aplicaciones relacionadas con los árboles binarios y que desarrolle sus habilidades de trabajo en equipo y programación orientada a objetos. 
\par\vspace{1.5cm}

\section*{\textcolor{red}{\textbf{INTRODUCCIÓN}}}



\subsection*{\textcolor{blue}{Algoritmos de inserción y eliminación (para un árbol binario balanceado AVL)}}

\subsection*{\textcolor{blue}{Algoritmo para la construcción de un árbol de expresión aritmética. }}

\subsection*{\textcolor{blue}{Algoritmos de árbol red black }}





\section*{\textcolor{red}{\textbf{ANÁLISIS DEL DESARROLLO DEL PROGRAMA}}}

\subsection*{\textcolor{blue}{  }}

\subsection*{\textcolor{blue}{  }}

\subsection*{\textcolor{blue}{  }}





\section*{\textcolor{red}{\textbf{CONCLUSIONES INDIVIDUALES}}}

\subsection*{\textcolor{blue}{  }}

\subsection*{\textcolor{blue}{Vázquez Flores José Angel}}



\subsection*{\textcolor{blue}{Zarate Menes Quetzalli}}


\end{document}